\documentclass[11pt, parskip=half]{scrartcl}

\usepackage{azuma}
\usepackage[raggedbottom]{stablespacing}
\usepackage[a4paper, margin=1in]{geometry} % center margins
\usepackage{needspace}
\usepackage{enumitem}
\usepackage{booktabs}
\usepackage{pgfplots}
\pgfplotsset{compat=newest}
\setlist[itemize]{topsep=0pt, itemsep=0.1em} %partopsep, parsep

\begin{document}
    \title{Gram Schmidt process}
    \subtitle{Notes}
    \maketitle
    \tableofcontents

    \section{Prerequisites}
    \begin{definition}[Linearly independent vectors]
        A set of vectors $\{v_1, v_2, \dots, v_k\}$ is linearly independent if the only solution to the equation
        $c_1 v_1 + c_2 v_2 + \dots + c_k v_k = 0$ is $c_1 = c_2 = \dots = c_k = 0$. 
        
        Intuitively this means that no vector in the set can be expressed as a linear combination of the others.
    \end{definition}
    \begin{definition}[Vector basis]
        A set of vectors $\{v_1, v_2, \dots, v_k\}$ is a basis for a vector space if:
        \begin{itemize}
            \item It's linearly independent (each vector in the set is linearly independent of one another)
            \item It spans the vector space (every vector in the vector space can be expressed as a linear combination of the basis vectors)
        \end{itemize}
    \end{definition}
    \begin{theorem}
        If you have $n$ vectors in $\mathbb{F}^n$, place them as columns of an $n\times n$ matrix $A$. Then
        \[
        \det(A)\neq 0
        \iff
        \text{the vectors are linearly independent}
        \iff
        \text{they form a basis of }\mathbb{F}^n.
        \]
        If the number of vectors is not $n$, or if the vectors are not in $\mathbb{F}^n$, then this determinant
        criterion does not apply (since there is no associated square matrix).
    \end{theorem}
    \begin{definition}[Inner (dot) product for vectors over $\mathbb{R}$]
        The inner product of two vectors $v = (v_1, v_2, \dots, v_n)$ and $w = (w_1, w_2, \dots, w_n)$ is defined as:
        $\langle v, w \rangle = v_1 w_1 + v_2 w_2 + \dots + v_n w_n$, or the matrix form:
        $\langle v, w \rangle = v^T w$.
    \end{definition}
    \begin{definition}[Inner (dot) product for vectors over $\mathbb{C}$]
        The inner product of two vectors $v = (v_1, v_2, \dots, v_n)$ and $w = (w_1, w_2, \dots, w_n)$ is defined as:
        $\langle v, w \rangle = \bar{v_1} w_1 + \bar{v_2} w_2 + \dots + \bar{v_n} w_n$, or the matrix form:
        $\langle v, w \rangle = v^{\dagger} w$.
    \end{definition}
    \begin{definition}[Orthogonal vectors]
        A set of vectors $\{v_1, v_2, \dots, v_k\}$ is orthogonal if $\langle v_i, v_j \rangle = 0$ for all $i \neq j$.
    \end{definition}
    \begin{definition}[Orthonormal vectors]
        A set of vectors $\{v_1, v_2, \dots, v_k\}$ is orthonormal if it's orthogonal and each vector has unit norm.
    \end{definition}
    \begin{definition}[Norm (length) of a vector]
        The norm of a vector $v$ is defined as $\|v\| = \sqrt{\langle v, v \rangle}$.
    \end{definition}
    \begin{definition}[Projection of a vector onto another]
        The projection of a vector $v$ onto a vector $w$ is defined as:
        $\text{proj}_w v = \frac{\langle w, v \rangle}{\langle w, w \rangle} w \qquad (w\neq 0)$.
    \end{definition}
    \section{Gram Schmidt Process}
    \begin{definition}[Gram Schmidt process]
        Gram Schmidt process is a method for computing a set of orthogonal vectors from a set of linearly independent vectors.
    \end{definition}

    \noindent\textbf{Gram--Schmidt orthogonalization.}
    Given $k$ nonzero linearly independent vectors $\mathbf{v}_1,\ldots,\mathbf{v}_k$,
    the Gram--Schmidt process defines the vectors $\mathbf{u}_1,\ldots,\mathbf{u}_k$ as follows:

    \[
    \begin{aligned}
    \mathbf{u}_1 &= \mathbf{v}_1,\\
    \mathbf{u}_2 &= \mathbf{v}_2 - \operatorname{proj}_{\mathbf{u}_1}(\mathbf{v}_2),\\
    \mathbf{u}_3 &= \mathbf{v}_3 - \operatorname{proj}_{\mathbf{u}_1}(\mathbf{v}_3) - \operatorname{proj}_{\mathbf{u}_2}(\mathbf{v}_3),\\
    \mathbf{u}_4 &= \mathbf{v}_4 - \operatorname{proj}_{\mathbf{u}_1}(\mathbf{v}_4) - \operatorname{proj}_{\mathbf{u}_2}(\mathbf{v}_4) - \operatorname{proj}_{\mathbf{u}_3}(\mathbf{v}_4),\\
    &\ \ \vdots\\
    \mathbf{u}_k &= \mathbf{v}_k - \sum_{j=1}^{k-1}\operatorname{proj}_{\mathbf{u}_j}(\mathbf{v}_k),
    \end{aligned}
    \qquad
    \begin{aligned}
    \mathbf{e}_1 &= \frac{\mathbf{u}_1}{\lVert \mathbf{u}_1\rVert},\\
    \mathbf{e}_2 &= \frac{\mathbf{u}_2}{\lVert \mathbf{u}_2\rVert},\\
    \mathbf{e}_3 &= \frac{\mathbf{u}_3}{\lVert \mathbf{u}_3\rVert},\\
    \mathbf{e}_4 &= \frac{\mathbf{u}_4}{\lVert \mathbf{u}_4\rVert},\\
    &\ \ \vdots\\
    \mathbf{e}_k &= \frac{\mathbf{u}_k}{\lVert \mathbf{u}_k\rVert}.
    \end{aligned}
    \]

    The sequence $\mathbf{u}_1,\ldots,\mathbf{u}_k$ is the required system of orthogonal vectors,
    and the normalized vectors $\mathbf{e}_1,\ldots,\mathbf{e}_k$ form an orthonormal set.
    The calculation of the sequence $\mathbf{u}_1,\ldots,\mathbf{u}_k$ is known as Gram--Schmidt
    orthogonalization, and the calculation of the sequence $\mathbf{e}_1,\ldots,\mathbf{e}_k$ is
    known as Gram--Schmidt orthonormalization.

\end{document}